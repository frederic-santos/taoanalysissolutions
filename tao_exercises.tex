\documentclass[11pt]{article}
\title{Propositions of solutions for \textit{Analysis I} by Terence Tao}
\author{Frédéric Santos}
% General packages:
\usepackage{a4wide}
\usepackage[english]{babel}
\usepackage[utf8]{inputenc} 
\usepackage[T1]{fontenc}
\usepackage{titlesec}
\titlelabel{\thetitle.\quad}
% Fonts and math packages:
\usepackage{lmodern}
\usepackage{amsmath}
\usepackage[matha,mathb]{mathabx}
\usepackage{mbboard}
% Macros:
\newcommand{\successor}[1]{#1 \! +\!\!+}
% Environment:
\newenvironment{exo}[2]{\textsc{Exercise #1}. --- \textit{#2} \\}

% Begin doc:
\begin{document}
\maketitle

\section{Introduction}
\label{sec:introduction}
No exercises in this chapter.

\section{The natural numbers}
\label{sec:natural-numbers}
\begin{exo}{2.2.1}{Prove that the addition is associative, i.e. that
    for any natural numbers $a, b, c$, we have $(a+b) + c = a + (b+c)$.}

  Let's use induction on $c$ while keeping $a$ and $b$ fixed.
  
  \begin{itemize}
  \item Base case for $c=0$: let's prove that $(a+b)+0 = a+(b+0)$. The
    left hand side is equal to $(a+b)$ according to Lemma 2.2.3. For the
    right hand side, if we apply the same lemma to the $(b+0)$ part,
    we get $a + (b+0) = a+b$. Both sides are equal to $a+b$, and the
    base case is thus done.
  \item Now let's suppose inductively that $(a+b) + c = a + (b+c)$: we
    have to prove that $(a+b) + \successor{c} = a + (b +
    \successor{c})$. Using Lemma 2.2.3 on the right
    hand side leads to $a + \successor{(b+c)}$. Now consider the left
    hand side. Using still the same lemma, we get $(a+b) +
    \successor{c} = \successor{\left((a+b) + c\right)}$. By the
    inductive hypothesis, this is also equal to
    $\successor{\left(a + (b+c)\right)}$. And, using the lemma 2.2.3
    again, this also leads to $a + \successor{b+c}$. Therefore, both
    sides are equal to $a + \successor{b+c}$, and we have closed the induction.
  \end{itemize}
\end{exo}
\end{document}
