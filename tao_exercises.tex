\documentclass[11pt]{article}
\title{Propositions of solutions for \textit{Analysis I} by Terence Tao}
\author{Frédéric Santos}
% General packages:
\usepackage{a4wide}
\usepackage[english]{babel}
\usepackage[utf8]{inputenc} 
\usepackage[T1]{fontenc}
\usepackage{titlesec}
\titlelabel{\thetitle.\quad}
\usepackage{enumitem}
% Fonts and math packages:
\usepackage{lmodern}
\usepackage{amsmath}
\usepackage[matha,mathb]{mathabx}
\usepackage{mbboard}
% Macros:
\newcommand{\successor}[1]{#1 \! +\!\!\!+}
% Environment:
\newenvironment{exo}[2]{\noindent \textsc{Exercise #1}. --- \textit{#2} \\}

% Begin doc:
\begin{document}
\maketitle

\section{Introduction}
\label{sec:introduction}
No exercises in this chapter.

\section{The natural numbers}
\label{sec:natural-numbers}
\begin{exo}{2.2.1}{Prove that the addition is associative, i.e. that
    for any natural numbers $a, b, c$, we have $(a+b) + c = a + (b+c)$.}

  Let's use induction on $c$ while keeping $a$ and $b$ fixed.
  
  \begin{itemize}
  \item Base case for $c=0$: let's prove that $(a+b)+0 = a+(b+0)$. The
    left hand side is equal to $(a+b)$ according to Lemma 2.2.3. For the
    right hand side, if we apply the same lemma to the $(b+0)$ part,
    we get $a + (b+0) = a+b$. Both sides are equal to $a+b$, and the
    base case is thus done.
  \item Now let's suppose inductively that $(a+b) + c = a + (b+c)$: we
    have to prove that $(a+b) + \successor{c} = a + (b +
    \successor{c})$. Using Lemma 2.2.3 on the right
    hand side leads to $a + \successor{(b+c)}$. Now consider the left
    hand side. Using still the same lemma, we get $(a+b) +
    \successor{c} = \successor{\left((a+b) + c\right)}$. By the
    inductive hypothesis, this is also equal to
    $\successor{\left(a + (b+c)\right)}$. And, using the lemma 2.2.3
    again, this also leads to $a + \successor{b+c}$. Therefore, both
    sides are equal to $a + \successor{b+c}$, and we have closed the
    induction.
  \end{itemize}
\end{exo}

\begin{exo}{2.2.2}{Let $a$ be a positive number. Prove that there
    exists exactly one natural number $b$ such that $\successor{b} =
    a$.}
  
  Let's use induction on $a$.

  \begin{itemize}
  \item Base case for $a=1$: we know that $b=0$ matches this property,
    since $\successor{0} = 1$ by Definition 2.1.3. Furthermore, there
    is only one solution. Suppose that is another natural number $b$
    such that $\successor{b} = 1$. Then, we would have
    $\successor{b} = \successor{0}$, which would imply $b = 0$ by
    Axiom 2.4. The base case is demonstrated.
  \item Let's suppose inductively that there is exactly one natural
    number $b$ such that $\successor{b} = a$. We have to prove that
    there is exactly one natural number $b'$ such that
    $\successor{b'} = \successor{a}$. By the induction hypothesis, and
    taking $b' = \successor{b}$, we have
    $\successor{b'} = \successor{(\successor{b})} = \successor{a}$. So
    there exists a solution, with $b' = \successor{b} = a$. Uniqueness is
    given by Axiom 2.4.: if $\successor{b'} = \successor{a}$, then we
    necessarily have $b'=a$.
  \end{itemize}
\end{exo}

\begin{exo}{2.2.3}{Let $a, b, c$ be natural numbers. Prove the
    following properties of order for natural numbers:}
  \begin{enumerate}[label=\emph{(\alph*)}]
  \item Reflexivity: $a \geq a$. This is true since $a = 0 + a$ by
    Definition 2.2.1. By commutativity of addition, we can also write
    $a = a + 0$. So there is indeed a natural number $n$ (with
    $n=0$) such that $a = a + n$, i.e. $a \geq a$.
  \item Transitivity: if $a \geq b$ and $b \geq c$, then $a \geq
    c$. From the part $a \geq b$, there exists a natural number $n$
    such that $a = b + n$ according to Definition 2.2.11. A similar
    consideration for the part $b \geq c$ leads to $b = c + m$, $m$
    being a natural number. Combining together those two equalities,
    we can write $a = b + n = (c + m) + n = c + (m + n)$ by
    associativity (see Exercise 2.2.1). Then, $n+m$ being a natural
    number\footnote{This is a trivial induction from the definition of
      addition.}, the transitivity is demonstrated.
  \item Anti-symmetry: if $a \geq b$ and $b \geq a$, then $a=b$. From
    the part $a \geq b$, there exists a natural number $n$ such that $a
    = b + n$. Similarly, there exists a natural number $m$ such that $b
    = a + m$. Combining those two equalities leads to $a = b + n = (a +
    m) + n = a + (m + n)$. By cancellation law (Proposition 2.2.6), we
    can conclude that $0 = m + n$. According to Corollary 2.2.9, this
    leads to $m = n = 0$. Therefore, both $m$ and $n$ are null,
    meaning that $a = b + 0 = b$.
  \item Preservation of order: $a \geq b$ iff $a + c \geq b +
    c$. First, let's prove that
    $a+c \geq b+c \Longrightarrow a \geq b$. If $a+c \geq b+c$, there
    exists a natural number $n$ such that $a+c = b+c+n$. By cancellation
    law (Proposition 2.2.6)\footnote{And also associativity and
      commutativity that we do not detail explicitly here.}, we
    conclude that $a = b + n$, i.e. $a\geq b$, thus demonstrating the
    first implication. Conversely, let's suppose that $a \geq
    b$. There exists a natural number $m$ such that $a = b +
    m$. Therefore, $a + c = b+m+c$ for any natural number
    $c$\footnote{It is easy to demonstrate that, if $a=b$, then
      $a+n=b+n$ for any natural numbers $a,b,n$. This is a trivial
      induction on $n$, but it seems to me that it should be proved. I
      am not totally sure that we can use that starting only from
      Peano's axioms.}. Still by associativity and commutativity, we
    can rewrite this as $a+c = (b+c) +m$, i.e. $a+c \geq b+c$.
  \item $a < b$ iff $\successor{a} \leq b$. First, let's prove that
    $\successor{a} \leq b \Longrightarrow a < b$. By definition of
    ordering, there exists a natural number $n$ such that
    $b = (\successor{a}) + n$. By definition of addition, we can
    re-write: $b = \successor{(\successor{a} + n)}$. Then, by
    commutativity and yet again by definition of addition,
    $b = \successor{(n + \successor{a})} = (\successor{n}) +
    (\successor{a})$. Thus, there exists a natural number
    $\successor{n}$ such that $b = \successor{n} + a$, which means that
    $b \geq a$. But we still have to prove that $a \neq b$. Let's
    suppose that $a=b$: in this case, by cancellation law, we would
    have $\successor{n} = 0$, which is impossible according to Axiom
    2.3 (0 is not the successor of any natural number). Thus, $a \neq
    b$ et $b \geq a$: we have showed that $a < b$. 

    Conversely, let's prove that
    $a < b \Longrightarrow \successor{a} \leq b$. Starting from that
    strict inequality, there exists a \textit{positive}\footnote{We
      make use here of the statement \textit{(f)} demonstrated
      below. There is no circularity here, since proving \textit{(f)}
      will not make use of \textit{(e)}.} natural number $n$ such that
    $b = a + n$. By Lemma 2.2.10, since $n$ is positive, it has one
    unique antecessor $m$, so that $n$ can be written
    $\successor{m}$. Thus,
    $b = a + (\successor{m}) = \successor{(a + m)} = \successor{(m +
      a)} = m + (\successor{a}) = (\successor{a}) + m$. And, $m$ being
    a natural number, this corresponds to the statement $b \geq a$.
  \item $a < b$ iff $b=a+d$ for some positive number $d$. First, let's
    prove the first implication, $a<b \Longrightarrow b=a+d$ with
    $d \neq 0$. Since $a<b$, we have in particular $a \leq b$, and
    there exists a natural number $d$ such that $b=a+d$. For the sake
    of contradiction, let's suppose that $d=0$. We would have $b=a$,
    which would contradict the condition $a\neq b$ of the strict
    inequality. Thus, $d$ is a positive number, which demonstrates the
    left-to-right implication.

    Conversely, let's suppose that $b = a+d$, with $d \neq 0$. This
    expression gives immediately $a \leq b$. But if $a=b$, by
    cancellation law, this would lead to $0=d$, a contradiction with
    the fact that $d$ is a positive number. Thus, $a\neq b$ and $a
    \leq b$, which demonstrates $a<b$.
  \end{enumerate}
\end{exo}
\end{document}
