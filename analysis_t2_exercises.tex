\documentclass[11pt]{article}
\title{Propositions of solutions for \textit{Analysis II} by Terence Tao}
\author{Frédéric Santos}
% General packages:
\usepackage{a4wide}
\usepackage[english]{babel}
\usepackage[utf8]{inputenc} 
\usepackage[T1]{fontenc}
\usepackage{titlesec}
\titlelabel{\thetitle.\quad}
\usepackage{enumitem}
% Fonts and math packages:
\usepackage{lmodern}
\usepackage{amsmath}
\numberwithin{equation}{section}
\usepackage[matha,mathb]{mathabx}
\usepackage{mbboard}
\usepackage{stmaryrd}
\usepackage{hyperref}
% Macros:
\newcommand{\successor}[1]{#1 \! +\!\!\!+}
\newcommand{\aval}[1]{\left\lvert #1 \right\rvert}
\newcommand{\intset}[2]{\llbracket #1, #2 \rrbracket}
\newcommand{\inv}[1]{#1^{-1}}
\newcommand{\rr}{\mathbb{R}}
\newcommand{\partsof}[1]{\mathcal{P}\left( #1 \right)}
\newcommand{\minus}{\, \textrm{---}\!\textrm{--} \:}
\newcommand{\quot}{\, \textrm{/}\!\textrm{/} \:}
\newcommand{\nn}{\mathbb{N}}
\newcommand{\zz}{\mathbb{Z}}
\newcommand{\qq}{\mathbb{Q}}
\newcommand{\floor}[1]{\lfloor #1 \rfloor}
\newcommand{\formallimit}[1]{\text{LIM}_{n \to \infty} #1}
\newcommand{\seq}[2]{(#1)_{n=#2}^\infty}
\newcommand{\limit}[1]{\text{lim}_{n \to \infty} #1}
\newcommand{\extrr}{\overline{\rr}}
\newcommand{\adh}[1]{\overline{#1}}
\newcommand{\liminfp}[2]{\inf (#1^+_N)_{N=#2}^{\infty}}
\newcommand{\abs}[1]{\left\lvert #1 \right\rvert}
\renewcommand{\P}{\mathbf{P}}
\newcommand{\Q}{\mathbf{Q}}
\renewcommand{\epsilon}{\varepsilon}
\newcommand{\infint}[2]{\underline{\int}_{#2} \, #1}
\newcommand{\supint}[2]{\overline{\int}_{#2} \, #1}
% Lemmas:
\usepackage{amsthm}
\newtheorem*{lem}{Lemma}
\newtheorem*{theorem}{Theorem}
% Environment:
\newenvironment{exo}[2]{\noindent \textsc{Exercise #1}. ---
  \textit{#2} \vspace{3mm}}

%%%%%%%%%%%%%%
%%% Begin doc:
\begin{document}
\maketitle
\tableofcontents

\vskip 15mm

\noindent \textbf{Remarks.} The numbering of the Exercises follows the
fourth edition of \textit{Analysis II}. In order to make the
references to \textit{Analysis I} easier, we consider that we begin
with Chapter 12 here, as in earlier editions of the textbook. Thus, in
particular, a reference to ``Exercise 4.3.3'' (for instance) will
always mean ``Exercise 4.3.3 from \textit{Analysis I}''.

\pagebreak
\setcounter{section}{11}
\section{Metric spaces}
\label{sec:metric-spaces}
\begin{exo}{12.1.1}{Prove Lemma 12.1.1}

  Consider the sequence $\seq{a_n}{m}$ defined by
  $a_n := d(x_n, x) = |x_n - x|$ for all $n \geq m$. We have to prove
  that $\lim_{n \to \infty} a_n = 0$ if and only if
  $\lim_{n \to \infty} x_n = x$.

  \begin{itemize}
  \item Let be $\epsilon > 0$. If $\lim_{n \to \infty} a_n = 0$, then
    there exists an $N \geq m$ such that $|a_n| < \epsilon$ whenever
    $n \geq N$. Thus, there exists an $N \geq m$ such that
    $|x_n - x| < \epsilon$ whenever $n \geq N$, which means that
    $\lim_{n \to \infty} x_n = x$.
  \item Let be $\epsilon > 0$. Conversely, if $\lim_{n \to \infty} x_n
    = x$, then there exists an $N \geq m$ such that $|x_n - x| <
    \epsilon$ whenever $n \geq N$. But since $|a_n| := |x_n - x|$, it
    means that $\lim_{n \to \infty} a_n = 0$, as expected.
  \end{itemize}
\end{exo}

\begin{exo}{12.1.2}{Show that the real line with the metric $d(x, y) :=
    |x-y|$ is indeed a metric space.}

  Using Proposition 4.3.3, this claim is obvious. All claims (a)--(d)
  of Definition 12.1.2 are satisfied because:
  \begin{enumerate}[label=(\alph*)]
  \item comes from Proposition 4.3.3(e)
  \item also comes from Proposition 4.3.3(e)
  \item comes from Proposition 4.3.3(f)
  \item comes from Proposition 4.3.3(g).
  \end{enumerate}
\end{exo}

\begin{exo}{12.1.3}{Let $X$ be a set, and let
    $d : X \times X \to [0, \infty)$ be a function. With respect to
    Definition 12.1.2, give an example of a pair $(X,d)$ which...
    \vskip -4mm}
  
  \begin{enumerate}[label=(\alph*)]
  \item obeys the axioms (bcd) but not (a).

    Consider $X = \rr$, and $d$ defined by $d(x,x) = 1$ and $d(x,y) =
    5$ for all $x \neq y \in \rr$.
  \item obeys the axioms (acd) but not (b).

    Consider $X = \rr$, and $d$ defined by $d(x,y) = 0$ for all $x, y
    \in \rr$.
  \item obeys the axioms (abd) but not (c).

    Consider $X = \rr$, and $d$ defined by $d(x,y) = \max(x-y,0)$ for
    all $x,y \in \rr$.
  \item obeys the axioms (abc) but not (d).

    Consider the finite set $X := \{1,2,3\}$ and the application $d$
    defined by $d(1,2) = d(2,1) = d(2,3) = d(3,2) := 1$, and
    $d(1,3) = d(3,1) := 5$, and $d(x,x) = 0$ for all $x \in X$.
  \end{enumerate}  
\end{exo}
\end{document}
%%% Local Variables:
%%% mode: latex
%%% TeX-master: t
%%% End:
